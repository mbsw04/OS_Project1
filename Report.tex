\documentclass[12pt, onecolumn]{IEEEtran}
\usepackage{graphicx}
\usepackage{hyperref}
\usepackage{listings}
\usepackage{amsmath}
\usepackage{cite}
\usepackage{geometry}
\geometry{a4paper, margin=1in}

\title{Inter-Process Communication in Air Traffic Control}
\author{Your Name \\ Course Section \\ NETID (e.g., jdoe3@students.kennesaw.edu)}
\date{\today}

\begin{document}

\maketitle

\begin{abstract}
This report presents the implementation of an inter-process communication (IPC) system for an Air Traffic Control (ATC) simulation. The objective is to demonstrate IPC using named pipes (FIFOs) to facilitate communication between an ATC tower and an aircraft. The project highlights key concepts of process synchronization, data flow between independent processes, and real-world applications of IPC in aviation systems.
\end{abstract}

\section{Introduction}
Inter-process communication (IPC) enables separate processes to exchange data, which is crucial in real-time systems like ATC. In our implementation:
\begin{itemize}
    \item The ATC tower sends landing instructions to an aircraft via a named pipe.
    \item The aircraft receives the message, acknowledges it, and responds.
    \item The ATC tower reads the acknowledgment to ensure safe landing.
\end{itemize}

\section{Implementation Details}

\subsection{Overview of IPC in Air Traffic Control}
Named pipes (FIFOs) allow unidirectional or bidirectional communication between processes. In this project, two FIFOs are used:
\begin{itemize}
    \item \texttt{atc\_pipe}: The ATC tower writes landing clearance messages.
    \item \texttt{aircraft\_pipe}: The aircraft responds with an acknowledgment.
\end{itemize}

\subsection{ATC Tower Code}
The ATC tower writes a message to the named pipe and then waits for an acknowledgment:
\begin{lstlisting}[language=C++, caption=ATC Tower Sending Message]
#include <iostream>
#include <fstream>
#include <unistd.h>

int main() {
    std::ofstream atcOut("atc_pipe");
    if (!atcOut) {
        std::cerr << "Error opening ATC pipe!" << std::endl;
        return 1;
    }
    atcOut << "ATC Tower: Permission granted to land." << std::endl;
    atcOut.close();
    sleep(2);

    std::ifstream aircraftIn("aircraft_pipe");
    if (!aircraftIn) {
        std::cerr << "Error opening Aircraft pipe!" << std::endl;
        return 1;
    }
    std::string response;
    getline(aircraftIn, response);
    aircraftIn.close();

    std::cout << "Received from aircraft: " << response << std::endl;
    return 0;
}
\end{lstlisting}

\subsection{Aircraft Code}
The aircraft reads the ATC instruction and sends an acknowledgment back:
\begin{lstlisting}[language=C++, caption=Aircraft Receiving and Responding]
#include <iostream>
#include <fstream>
#include <unistd.h>

int main() {
    std::ifstream atcIn("atc_pipe");
    if (!atcIn) {
        std::cerr << "Error opening ATC pipe!" << std::endl;
        return 1;
    }
    std::string message;
    getline(atcIn, message);
    atcIn.close();

    std::cout << "Received from ATC: " << message << std::endl;

    std::ofstream aircraftOut("aircraft_pipe");
    if (!aircraftOut) {
        std::cerr << "Error opening Aircraft pipe!" << std::endl;
        return 1;
    }
    aircraftOut << "Aircraft: Acknowledged. Landing now." << std::endl;
    aircraftOut.close();

    return 0;
}
\end{lstlisting}

\section{Environment Setup and Tool Usage}

\subsection{Setting Up Docker Environment}
1. Install Docker:
\begin{lstlisting}[language=bash]
sudo apt update && sudo apt install docker.io
\end{lstlisting}

2. Create a container with Ubuntu:
\begin{lstlisting}[language=bash]
docker run -it --name atc_env ubuntu bash
\end{lstlisting}

3. Install required tools inside the container:
\begin{lstlisting}[language=bash]
apt update && apt install g++ make vim
\end{lstlisting}

4. Verify installation:
\begin{lstlisting}[language=bash]
g++ --version && make --version
\end{lstlisting}

\section{Challenges and Solutions}
\begin{itemize}
    \item \textbf{Synchronization Issues}: Used sleep delays to ensure correct message order.
    \item \textbf{Named Pipe Permissions}: Ensured correct read/write permissions with \texttt{chmod 666}.
    \item \textbf{Blocking Read}: Used non-blocking techniques to prevent indefinite waiting for data.
    \item \textbf{Debugging IPC}: Used logs and intermediate file outputs to trace message flow.
\end{itemize}

\section{Results and Outcomes}
The ATC successfully communicates with the aircraft using named pipes, demonstrating IPC principles. The response times are minimal, and the system simulates real-world ATC scenarios effectively. Performance tests confirmed efficient message exchange with low overhead.

\section{Reflection and Learning}
This project enhanced understanding of IPC mechanisms, process synchronization, and system-level programming in Linux environments. The use of Docker streamlined environment setup, making it easier to test and deploy IPC-based applications.

\section{References}
\begin{thebibliography}{9}
    \bibitem{linuxpipes} Linux Pipes Documentation, \url{https://man7.org/linux/man-pages/man7/pipe.7.html}
    \bibitem{cpp} C++ Standard Library Documentation, \url{https://en.cppreference.com/w/}
    \bibitem{docker} Docker Documentation, \url{https://docs.docker.com/}
    \bibitem{atcsim} Air Traffic Control Simulation Studies, \url{https://www.faa.gov/}
\end{thebibliography}

\end{document}
